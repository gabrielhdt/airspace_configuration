\documentclass[twoside,twocolumn]{article}

\usepackage{fontspec}
\usepackage{polyglossia}
\usepackage{cite}
\usepackage{hyperref}


\usepackage[hmarginratio=1:1,top=32mm,columnsep=20pt]{geometry}
\usepackage[hang, small,labelfont=bf,up,textfont=it,up]{caption}
\usepackage{booktabs}

\usepackage{lettrine}

\usepackage{enumitem}
\setlist[itemize]{noitemsep}

\usepackage{abstract}
\renewcommand{\abstractnamefont}{\normalfont\bfseries} % Set the "Abstract" text to bold
\renewcommand{\abstracttextfont}{\normalfont\small\itshape} % Set the abstract itself to small italic text

\usepackage{titlesec} % Allows customization of titles
\renewcommand\thesection{\Roman{section}} % Roman numerals for the sections
\renewcommand\thesubsection{\roman{subsection}} % roman numerals for subsections
\titleformat{\section}[block]{\large\scshape\centering}{\thesection.}{1em}{} % Change the look of the section titles
\titleformat{\subsection}[block]{\large}{\thesubsection.}{1em}{} % Change the look of the section titles

\usepackage{fancyhdr} % Headers and footers
\pagestyle{fancy} % All pages have headers and footers
\fancyhead{} % Blank out the default header
\fancyfoot{} % Blank out the default footer
\fancyhead[C]{Running title $\bullet$ May 2016 $\bullet$ Vol. XXI, No. 1} % Custom header text
\fancyfoot[RO,LE]{\thepage} % Custom footer text

\usepackage{titling} % Customizing the title section

\usepackage{hyperref} % For hyperlinks in the PDF

\title{Airspace configuration decision based on Monte-Carlo tree search}
\author{Gabriel~Hondet, Beno\^{\i}t Viry}
\date{\today}

%----------------------------------------------------------------------------------------
%	TITLE SECTION
%----------------------------------------------------------------------------------------

\setlength{\droptitle}{-4\baselineskip} % Move the title up

\pretitle{\begin{center}\Huge\bfseries} % Article title formatting
\posttitle{\end{center}} % Article title closing formatting

\renewcommand{\maketitlehookd}{%
\begin{abstract}

\end{abstract}
}

%----------------------------------------------------------------------------------------

\begin{document}

\maketitle

\section{Introduction}
The airspace is divided into sectors, themselves divided into elementary modules.
Currently, configuration is mainly decided on the fly by chief officer (TODO).
This decision is based on the actual workload on each position. This intuitive
approach does not take into account future traffic. This papers aims at
providing a method offering a smoother workload during the day.

Several methods have been considered to solve the dynamic airspace configuration
problem, for instance via genetic algorithms in~\cite{sergeeve2017dynamic},
constraint local search in~\cite{jagare2013constraint}, integer linear
programming in~\cite{treimuth2016branch} or dynamic programming
in~\cite{bloem2010dynamic}.

To achieve that, as seen in~\cite{treimuth2016branch}
or~\cite{sergeeva2017dynamic}, a temporal sequence of configurations will be
provided. Two costs will be considered to create the sequence, namely the cost
associated to each configuration and transition. The former is based on the
workload estimated for one sector given a neural network
(see~\cite{gianazza2010forecasting}).



\bibliography{article}
\bibliographystyle{plain}
\end{document}
