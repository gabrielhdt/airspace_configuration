\documentclass{article}
\usepackage{fontspec}
\usepackage{polyglossia}
\usepackage{cite}
\usepackage{hyperref}
\usepackage{unicode-math}
\usepackage{amsmath,amsthm,stmaryrd}
\usepackage{algorithm,algpseudocode}

\title{Airspace configuration with Monte Carlo tree search}
\author{Us}
\date{\today}


\begin{document}
\maketitle

\section{Preliminaries}
\subsection{Base theory: the environment and notations}
Notations based on Markov Decision Problem's theory (taken
from~\cite{browne2012survey})
\begin{itemize}
  \item \(\mathcal{S}\) a set of states,
  \item \(\mathcal{A}\) a set of actions,
  \item \(T(s, a, s')\) a transition model that determines the probability of
      reaching state \(s'\) if action \(a\) is applied to state \(s\),
  \item \(R(s)\) a reward function
\end{itemize}

A policy is a mapping \(\pi \colon \mathcal{S} \to \mathcal{A}\). The aim is to
output, from a state \(s\), the action that gives the highest reward.

\subsection{Monte carlo methods~\cite{browne2012survey}}
We denote by \(Q(s, a)\) the expected reward of an action. Let
\begin{itemize}
  \item \(N(s, a)\) be
    the number of times action \(a\) has been selected from state \(s\),
  \item \(N(s)\) the number of times the game has been played from state \(s\),
  \item \(\mathbb{I}_i(s, a)\) is 1 if action \(a\) was selected from state
    \(s\) on the \(i\)th playout from state \(s\) else 0 and
  \item \(z_i\) is the result of the \(i\)th simulation played.
\end{itemize}
In Monte Carlo methods, we have
\begin{equation}
  Q(s, a) = \frac{1}{N(s, a)}\sum_{i = 1}^{N(s)}\mathbb{I}_i(s, a)z_i
\end{equation}

\subsection{Bandits methods~\cite{browne2012survey},~\cite{kocsis2006bandit}}

\subsection{Monte Carlo tree search}
In the MCTS, the action \(a\) is the path that leads from the root to the best
node computed so far. To each node should correspond a state \(s\). This way, an
action \(a\) from a node with state \(s\) gives an other node with an other
state \(s'\).

\section{Algorithms}
\subsection{Monte carlo}
\begin{algorithm}
  \caption{General MCTS approach~\cite{browne2012survey}}
  \begin{algorithmic}
    \Function{MctsSearch}{$s_0$}
    \State{}create root node \(v_0\) with state \(s_0\)
    \While{within computational budget}
    \State{} \(v_l \gets\) \Call{TreePolicy}{$v_0$}
    \State{} \(\Delta \gets\) \Call{DefaultPolicy}{$s(v_l)$}
    \State{} \Call{Backup}{$v_l$, $\Delta$}
    \EndWhile{}
    \Return{\(a(\)\Call{BestChild}{$v_0$}\()\)}
    \EndFunction{}
  \end{algorithmic}
\end{algorithm}

\section{Per article review}
\paragraph{Branch and price~\cite{treimuth2016branch}}
The most important in this article is the model used. The rest depends heavily
on the algorithm used, which is the branch and price, a variant of the branch
and bound.

Three things to minimise:
\begin{itemize}
  \item atc workload \(a\),
  \item workload difference between atc \(b\),
  \item changes between consecutive configurations \(c\).
\end{itemize}
The objective function is therefore a linear combination \(\alpha a + \beta b +
\gamma c\).

The airspace is modelled by a weighted graph (with \(T\) the discretised time)
\(G = (V, E, W_t, t\in T)\), with
\begin{itemize}
  \item \(V\), the set of vertex, elementary sectors of the airspace weighted by
    its overload,
  \item \(E\), the set of edges, they connect two contiguous elementary sectors,
    they are weighted by the coordination workload for the planes crossing the
    border if the two sectors belong to two different controllers,
  \item \(W_t\) the weights at time \(t\in T\).
\end{itemize}


Three sets of binary decisions are introduced, controlling
\begin{itemize}
  \item for each feasible configuration, whether it is applied,
  \item for each edge, whether it is a frontier,
  \item for each edge, whether its frontier status has changed (shifts from
    frontier to not frontier and vice versa).
\end{itemize}
In each set, there is, per time period, one variable per feasible configuration
for the first one and one variable per edge for the second and third ones.

The way to generate feasible configurations isn't explained though\dots




\bibliography{article}
\bibliographystyle{plain}
\end{document}
